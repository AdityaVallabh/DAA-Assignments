%%%%%%%%%%%%%%%%%%%%%%%%%%%%%%%%%%%%%%%%%%%%%%%%%%%%%%%%%%%%%%%%%%%%%%%%%%%%%%%%
%2345678901234567890123456789012345678901234567890123456789012345678901234567890
%        1         2         3         4         5         6         7         8

\documentclass[letterpaper, 11 pt, conference]{ieeeconf}  % Comment this line out
                                                          % if you need a4paper
%\documentclass[a4paper, 10pt, conference]{ieeeconf}      % Use this line for a4
\usepackage{graphicx}
\graphicspath{ {images/} }% paper
\usepackage{amsmath}
\usepackage{algorithm}
\usepackage{array}
\usepackage{tabu}
\usepackage{siunitx}
\usepackage{booktabs}% for better rules in the table
%\usepackage[ruled]{algorithm2e}

\usepackage[noend]{algpseudocode}
\makeatletter
\def\BState{\State\hskip-\ALG@thistlm}
\makeatother
\IEEEoverridecommandlockouts                              % This command is only
                                                          % needed if you want to
                                                          % use the \thanks command
\overrideIEEEmargins
% See the \addtolength command later in the file to balance the column lengths
% on the last page of the document

\title{\LARGE 
IDAA432C Assignment-3 (Group: 41)
}

\author{
  Aditya Vallabh\\
  \texttt{IIT2016517}
  \and
  Kiran Velumuri\\
  \texttt{ITM2016009}
  \and
  Neil Syiemlieh\\
  \texttt{IIT2016125}
  \and
  Tauhid Alam\\
  \texttt{BIM2015003}
}


\begin{document}



\maketitle
\thispagestyle{empty}
\pagestyle{empty}


%%%%%%%%%%%%%%%%%%%%%%%%%%%%%%%%%%%%%%%%%%%%%%%%%%%%%%%%%%%%%%%%%%%%%%%%%%%%%%%%

\subsection*{ \textbf{Question} }
\textbf{Write an efficient algorithm to achieve the following. Given a set of numbers, find out the smallest and the largest numbers in the list. Swap the smallest with the first element and largest with the last element. Find the second smallest and second largest and swap them with their proper positions. Continue in this way until all the elements are sorted. Track the movement of every element in the list while sorting. Do the necessary experimentation and analysis with your algorithm.}


%%%%%%%%%%%%%%%%%%%%%%%%%%%%%%%%%%%%%%%%%%%%%%%%%%%%%%%%%%%%%%%%%%%%%%%%%%%%%%%%
\section{Introduction}
This problem requires us to sort a given array in the manner described in the question, while keeping track of the locations every element obtains as it moves around the array. The sort is a basic sorting scheme. Tracking the movement of the elements simply involves storing every new location that the elements obtain after they are swapped.

\section{Algorithm Design}
\subsection{Sorting}
The algorithm needed is similar to the selection sort algorithm, which selects a single element from an array and swaps it with an element at the beginning of the array.

This algorithm, however, selects two elements from an unsorted subarray --- the smallest and the largest ones --- and swaps the smallest with the first element and the largest with the last element. The same is repeated for every inner subarray. This effectively sorts the array in a manner similar to selection sort.

\subsection{Tracking Movement}
To keep track of the movement of the elements as they are swapped around in the array, we have associated a linked list with every array element. This linked list stores the indices that the corresponding element obtains as it moves through different locations of the array.

The head nodes of all the linked lists are stored in another array that runs parallel to the given input array.

After two elements are swapped, the corresponding head nodes in the second array are also swapped and a new node is appended to both their corresponding linked lists. This node contains the new index of that element and the iteration at which the swap happened.

\begin{algorithm}[H]
\caption{Sort Algorithm}\label{alg:sort}
\begin{algorithmic}
\State Input: $arr, heads, n$
\For {$i \gets 0\ to\ \frac{n}{2} - 1$}
	\State $mini = getMinIdx(arr, n, i)$
    \If {$mini \neq i$}
    	\State $swap(arr, heads, i+1, mini, i)$
    \EndIf
    \State $maxi = getMaxIdx(arr, n, i)$
    \If {$maxi \neq n-i-1$}
    	\State $swap(arr, heads, i+1, maxi, n-i-1)$
    \EndIf
\EndFor
\end{algorithmic}
\end{algorithm}

\begin{algorithm}[H]
\caption{Swap Algorithm}\label{alg:swap}
\begin{algorithmic}
\State Input: $arr, heads, it, a, b$
\State $heads[a] \gets insertNode(heads[a], it, b)$
\State $heads[b] \gets insertNode(heads[a], it, a)$
\State $tmp \gets arr[a]$
\State $arr[a] \gets arr[b]$
\State $arr[b] \gets tmp$
\State $temp \gets heads[a]$
\State $heads[a] \gets heads[b]$
\State $heads[b] \gets temp$
\end{algorithmic}
\end{algorithm}

\begin{algorithm}[H]
\caption{getMinIdx Algorithm}\label{alg:getMin}
\begin{algorithmic}
\State Input: $arr, n, i$
\State $idx \gets i$
\State $min \gets arr[i]$
\For {$j \gets i+1\ to\ n-i-1$}
	\If {$arr[j] < min$}
    	\State $min \gets arr[j]$
        \State $idx \gets j$
    \EndIf
\EndFor
\State $return~idx$
\end{algorithmic}
\end{algorithm}

\begin{algorithm}[H]
\caption{getMaxIdx Algorithm}\label{alg:getMax}
\begin{algorithmic}
\State Input: $arr, n, i$
\State $idx \gets i$
\State $max \gets arr[n-i-1]$
\For {$j \gets i\ to\ n-i-2$}
	\If {$arr[j] \geq max$}
    	\State $max \gets arr[j]$
        \State $idx \gets j$
    \EndIf
\EndFor
\State $return~idx$
\end{algorithmic}
\end{algorithm}

\section{Analysis and Discussion}
\subsection{Time Complexity}
Finding the smallest and the largest element in an array of size $n$ takes $O(n)$ time. The implemented sorting algorithm does this twice for every subarray from the $i\textsuperscript{th}$ to the $(n - i)\textsuperscript{th}$ element, where each subarray has $k$ elements. Hence, the total number of computations for an array of size $n$ is given by the following relation:
\begin{align*}
T(n) &= 2(O(n) + O(n-2) + ... + O(2)) \\
&= 2\sum_{k=0}^{\frac{n}{2} - 1} O(n - 2k) \\
&= \frac{n^2 + 2n}{2}
\end{align*}
This clearly has the time complexity of $O(n^2)$. This is the case for the best, average and worst cases. \\ 
\subsection{Space Complexity}
The algorithm as such doesn't require any additional space as it is an in-place sorting algorithm. However we are supposed to track the positions of each element while sorting. In order to store this additional information we require extra space. One naive method would be to store all intermediate arrays but this would unnecessarily waste a lot of space. So we have implemented linked lists where a node gets pushed to the respective list only when the position of the corresponding element in the array changes. Hence the space complexity is directly proportional to the number of swaps. \\ 
\textbf{Best Case:} When all the elements are already in a sorted fashion, no elements need to be swapped and hence the space complexity would be - $\Omega (n)$ \\ 
\textbf{Worst Case:} The maximum number of swaps that can occur is $n-1$. So then the complexity would be - $O(n + 2*(n-1)) = O(3*n - 2) = O(n)$. \\ 
\textbf{Average Case:} As in both cases we got the same complexity, the average case complexity would be $\theta (n)$
 \\ 
\subsection{Stability}
The sorting algorithm has to traverse the entire array to find the smallest and largest elements, by the $getMinIdx$ and $getMaxIdx$ functions, respectively. This would render it either stable or unstable depending on the condition by which the elements are compared within these functions.

Using the conditions $(arr[j] < min)$ in $getMinIdx$ and $(arr[j] \geq max)$ in $getMaxIdx$ ensures that the first occurrence of the smallest element and the last occurrence of the largest element are selected. This would make the implemented sorting algorithm a stable one. \\ 

\section{Experimental Study}
Following are the experimental findings after profiling the data and plotting the relevant graphs using $gnuplot$.

\begin{figure}[H]
\includegraphics[scale=0.65]{selectSort}
\caption{Time Complexity for the Sorting Algorithm}
\end{figure}

It can be seen that the derived expression for time complexity agrees with the obtained graph which is a semi-parabola denoting the graph (Fig. 1) of $y = kx^2$ with varying values of $k$ for the best, average and worst cases.

\begin{figure}[H]
\includegraphics[scale=0.65]{space1}
\caption{Space Complexity for the Sorting Algorithm}
\end{figure}

The graph for space complexity is linear thus agreeing with the derived expression. \\ 


\section{Conclusion}

In this paper we proposed an algorithm to sort an array of numbers in the given method. The time-complexity of our algorithm is $\theta (n^2)$. Moreover we have also tracked the positions of each element in the array by implementing linked lists with a space complexity of $\theta (n)$.

\end{document}
